\documentclass{article}

\usepackage{xcolor}
\usepackage{graphicx}
\usepackage{fancyvrb}
\usepackage{listings}
\usepackage[T1]{fontenc}
\usepackage{hyperref}
\usepackage{amsmath}

\definecolor{officegreen}{rgb}{0, 0.5, 0}
\definecolor{navy}{rgb}{0, 0, 0.5}
\definecolor{linecolor}{rgb}{0.5, 0.6875, 0.6875}
\definecolor{outputcolor}{rgb}{0.375, 0.375, 0.375}

\newcommand{\id}[1]{\textcolor{black}{#1}}
\newcommand{\com}[1]{\textcolor{officegreen}{#1}}
\newcommand{\inact}[1]{\textcolor{gray}{#1}}
\newcommand{\kwd}[1]{\textcolor{navy}{#1}}
\newcommand{\num}[1]{\textcolor{officegreen}{#1}}
\newcommand{\ops}[1]{\textcolor{purple}{#1}}
\newcommand{\prep}[1]{\textcolor{purple}{#1}}
\newcommand{\str}[1]{\textcolor{olive}{#1}}
\newcommand{\lines}[1]{\textcolor{linecolor}{#1}}
\newcommand{\fsi}[1]{\textcolor{outputcolor}{#1}}
\newcommand{\omi}[1]{\textcolor{gray}{#1}}

% Overriding color and style of line numbers
\renewcommand{\theFancyVerbLine}{
\lines{\small \arabic{FancyVerbLine}:}}

\lstset{%
  backgroundcolor=\color{gray!15},
  basicstyle=\ttfamily,
  breaklines=true,
  columns=fullflexible
}

\title{{page-title}}
\date{}

\begin{document}

\maketitle

\section*{API Documentation generation}



The \href{commandline.html}{command-line tool \texttt{fsdocs}} can be used to generate documentation
for F\# libraries with XML comments.  The documentation is normally built using \texttt{fsdocs build} and developed using \texttt{fsdocs watch}. For
the former the output will be placed in \texttt{output{\textbackslash}reference} by default.
\subsection*{Templates}



The HTML is built by instantiating a template. The template used is the first of:
\begin{itemize}
\item 

\texttt{docs/reference/\_template.html}

\item 

\texttt{docs/\_template.html}

\item 

The default template

\end{itemize}



Usually the same template can be used as for \href{content.html}{other content}.
\subsection*{Classic XML Doc Comments}



XML Doc Comments may use \href{https://docs.microsoft.com/en-us/dotnet/csharp/programming-guide/xmldoc/}{the normal F\# and C\# XML doc standards}.


The tags that form the core of the XML doc specification are:
\begin{Verbatim}[commandchars=\\\{\}]
{<}\id{c}{>}	{<}\id{para}{>}	{<}\id{see}{>}{*}	{<}\id{value}{>}
{<}\id{code}{>}	{<}\id{param}{>}{*}	{<}\id{seealso}{>}{*}	
{<}\id{example}{>}	{<}\id{paramref}{>}	{<}\id{summary}{>}	
{<}\kwd{exception}{>}{*}	{<}\id{permission}{>}{*}	{<}\id{typeparam}{>}{*}	
{<}\id{include}{>}{*}	{<}\id{remarks}{>}	{<}\id{typeparamref}{>}	
{<}\id{list}{>}	{<}\id{inheritdoc}{>}	{<}\id{returns}{>}	

\end{Verbatim}



In addition, you may also use the \href{https://github.com/fsharp/fslang-design/blob/master/tooling/FST-1031-xmldoc-extensions.md}{Recommended XML doc extensions for F\# documentation tooling}.
\begin{itemize}
\item 

\texttt{<a href = "...">} links

\item 

Arbitrary paragraph-level HTML such as \texttt{<b>} for bold in XML doc text

\item 

\texttt{<namespacedoc>} giving documentation for the enclosing namespace

\item 

\texttt{<exclude>} to exclude from XML docs

\item 

\texttt{<category>} to give a category for an entity or member. An optional \texttt{index} attribute can be specified
to help sort the list of categories.

\item 

\texttt{{\textbackslash}(...{\textbackslash})} for inline math and \texttt{\$\$...\$\$} and \texttt{{\textbackslash}[...{\textbackslash}]}for math environments, see \href{http://docs.mathjax.org.
Some}{http://docs.mathjax.org.
Some} escaping of characters (e.g. \texttt{<}, \texttt{>}) may be needed to form valid XML

\end{itemize}



An example of an XML documentation comment, assuming the code is in namespace \texttt{TheNamespace}:
\begin{Verbatim}[commandchars=\\\{\}]
\com{/// <summary>}
\com{///   Some actual comment}
\com{///   <para>Another paragraph, see  <see cref="T:TheNamespace.SomeType"/>. </para>}
\com{/// </summary>}
\com{///}
\com{/// <param name="x">The input</param>}
\com{///}
\com{/// <returns>The output</returns>}
\com{///}
\com{/// <example>}
\com{///   Try using}
\com{///   <code>}
\com{///      open TheNamespace}
\com{///      SomeModule.a}
\com{///   </code>}
\com{/// </example>}
\com{///}
\com{/// <namespacedoc>}
\com{///   <summary>A namespace to remember</summary>}
\com{///}
\com{///   <remarks>More on that</remarks>}
\com{/// </namespacedoc>}
\com{///}
\com{/// <category>Foo</category>}
\com{///}

\kwd{module} \ltyp{SomeModule} \ops{=} 
   \kwd{let} \lfun{someFunction} \lfun{x} \ops{=} \num{42} \ops{+} \lfun{x}

\com{/// <summary>}
\com{///   A type, see  <see cref="T:TheNamespace.SomeModule"/> and}
\com{///  <see cref="T:TheNamespace.SomeModule.someFunction"/>. </para>}
\com{/// </summary>}
\com{///}
\kwd{type} \ltyp{SomeType}{(}{)} \ops{=}
   \kwd{member} \lfun{x}{.}{P} \ops{=} \num{1}
\end{Verbatim}



Like types, members are referred to by xml doc sig.  These must currently be precise as the F\#
compiler doesn't elaborate these references from simpler names:
\begin{Verbatim}[commandchars=\\\{\}]
\kwd{type} \ltyp{Class2}{(}{)} \ops{=} 
    \kwd{member} \lfun{this}{.}{Property} \ops{=} \str{"more"}
    \kwd{member} \lfun{this}{.}\lfun{Method0}{(}{)} \ops{=} \str{"more"}
    \kwd{member} \lfun{this}{.}\lfun{Method1}{(}\lfun{c}{:} \ltyp{string}{)} \ops{=} \str{"more"}
    \kwd{member} \lfun{this}{.}\lfun{Method2}{(}\lfun{c}{:} \ltyp{string}{,} \lfun{o}{:} \ltyp{obj}{)} \ops{=} \str{"more"}

\com{/// <see cref="P:TheNamespace.Class2.Property" />}
\com{/// and <see cref="M:TheNamespace.Class2.OtherMethod0" />}
\com{/// and <see cref="M:TheNamespace.Class2.Method1(System.String)" />}
\com{/// and <see cref="M:TheNamespace.Class2.Method2(System.String,System.Object)" />}
\kwd{let} \lfun{referringFunction1} {(}{)} \ops{=} \str{"result"}
\end{Verbatim}



Generic types are referred to by .NET compiled name, e.g.
\begin{Verbatim}[commandchars=\\\{\}]
\kwd{type} \ltyp{GenericClass2}{<}{'}\id{T}{>}{(}{)} \ops{=} 
    \kwd{member} \lfun{this}{.}{Property} \ops{=} \str{"more"}

    \kwd{member} \lfun{this}{.}\lfun{NonGenericMethod}{(}\id{\_c}{:} {'}\id{T}{)} \ops{=} \str{"more"}

    \kwd{member} \lfun{this}{.}\lfun{GenericMethod}{(}\id{\_c}{:} {'}\id{T}{,} \id{\_o}{:} {'}\id{U}{)} \ops{=} \str{"more"}

\com{/// See <see cref="T:TheNamespace.GenericClass2`1" />}
\com{/// and <see cref="P:TheNamespace.GenericClass2`1.Property" />}
\com{/// and <see cref="M:TheNamespace.GenericClass2`1.NonGenericMethod(`0)" />}
\com{/// and <see cref="M:TheNamespace.GenericClass2`1.GenericMethod``1(`0,``0)" />}
\kwd{let} \lfun{referringFunction2} {(}{)} \ops{=} \str{"result"}

\com{(*}

\com{\#\# Go to Source links}

\com{'fsdocs' normally automatically adds GitHub links to each functions, values and class members for further reference.}

\com{This is normally done automatically based on the following settings:}

\com{    <RepositoryUrl>https://github.com/...</RepositoryUrl>}
\com{    <RepositoryBranch>...</RepositoryBranch>}
\com{    <RepositoryType>git</RepositoryType>}

\com{If your source is not built from the same project where you are building documentation then}
\com{you may need these settings:}

\com{    <FsDocsSourceRepository>...</FsDocsSourceRepository> -- the URL for the root of the source }
\com{    <FsDocsSourceFolder>...</FsDocsSourceFolder>         -- the root soure folder at time of build}

\com{It is assumed that `sourceRepo` and `sourceFolder` have synchronized contents.}

\com{\#\# Markdown Comments}

\com{You can use Markdown instead of XML in `///` comments. If you do, you should set `<UsesMarkdownComments>` in}
\com{your F\# project file.}

\com{> Note: Markdown Comments are not supported in all F\# IDE tooling.}

\com{\#\#\# Adding cross-type links to modules and types in the same assembly}

\com{You can automatically add cross-type links to the documentation pages of other modules and types in the same assembly.}
\com{You can do this in two different ways:}

\com{* Add a [markdown inline link](https://github.com/adam-p/markdown-here/wiki/Markdown-Cheatsheet\#links) were the link}
\com{title is the name of the type you want to link.}

\com{      /// this will generate a link to [Foo.Bar] documentation}

\com{* Add a [Markdown inline code](https://github.com/adam-p/markdown-here/wiki/Markdown-Cheatsheet\#code) (using}
\com{back-ticks) where the code is the name of the type you want to link.}

\com{      /// This will also generate a link to `Foo.Bar` documentation}

\com{You can use either the full name (including namespace and module) or the simple name of a type.}
\com{If more than one type is found with the same name the link will not be generated.}
\com{If a type with the given name is not found in the same assembly the link will not be generated.}
\com{*)}

\com{/// Contains two types [Bar] and [Foo.Baz]}
\kwd{module} \ltyp{Foo} \ops{=} 
    
    \com{/// Bar is just an `int` and belongs to module [Foo]}
    \kwd{type} \ltyp{Bar} \ops{=} \ltyp{int}
    
    \com{/// Baz contains a `Foo.Bar` as its `id`}
    \kwd{type} \ltyp{Baz} \ops{=} {\{} {id}{:} \ltyp{Bar} {\}}

    \com{/// This function operates on `Baz` types.}
    \kwd{let} \lfun{f} {(}\lfun{b}{:}\ltyp{Baz}{)} \ops{=} 
        \lfun{b}{.}\id{id} \ops{*} \num{42}

\com{/// Referencing [Foo3] will not generate a link as there is no type with the name `Foo3`}
\kwd{module} \ltyp{Foo3} \ops{=}
    
    \com{/// This is not the same type as `Foo.Bar`}
    \kwd{type} \ltyp{Bar} \ops{=} \ltyp{double}

    \com{/// Using the simple name for [Bar] will fail to create a link because the name is duplicated in }
    \com{/// [Foo.Bar] and in [Foo3.Bar]. In this case, using the full name works.}
    \kwd{let} \lfun{f2} \lfun{b} \ops{=}
         \lfun{b} \ops{*} \num{50}
\end{Verbatim}

\subsubsection*{Markdown Comments: Excluding APIs from the docs}



If you want to exclude modules or functions from the API docs you can use the \texttt{[omit]} tag.
It needs to be set on a separate tripple-slashed line, but it could be either the first or the last:
\begin{Verbatim}[commandchars=\\\{\}]
\com{/// [omit]}
\com{/// Some actual comment}
\kwd{module} \ltyp{Bar} \ops{=} 
   \kwd{let} \id{a} \ops{=} \num{42}
\end{Verbatim}

\subsection*{Building library documentation programmatically}



You can build library documentation programatically. To do this, load the assembly and open necessary namespaces:
\begin{Verbatim}[commandchars=\\\{\}]
\prep{\#r} \str{"FSharp.Formatting.ApiDocs.dll"}
\kwd{open} \id{FSharp}{.}\id{Formatting}{.}\id{ApiDocs}
\kwd{open} \id{System}{.}\id{IO}
\end{Verbatim}



Building the library documentation is easy - you just need to call
\texttt{ApiDocs.Generate} from your FAKE script or from F\# Interactive.
Assuming \texttt{root} is the root directory for your project, you can write:
\begin{Verbatim}[commandchars=\\\{\}]
\kwd{let} \id{file} \ops{=} \ltyp{Path}{.}\id{Combine}{(}\id{root}{,} \str{"bin/YourLibrary.dll"}{)}
\kwd{let} \id{input} \ops{=} \ltyp{ApiDocInput}{.}\id{FromFile}{(}\id{file}{)} 
\ltyp{ApiDocs}{.}\id{GenerateHtml}
    {(} {[} \id{input} {]}{,} 
      \lfun{output}\ops{=}\ltyp{Path}{.}\id{Combine}{(}\id{root}{,} \str{"output"}{)}{,}
      \lfun{collectionName}\ops{=}\str{"YourLibrary"}{,}
      \lfun{template}\ops{=}\ltyp{Path}{.}\id{Combine}{(}\id{root}{,} \str{"templates"}{,} \str{"template.html"}{)}{,}
      \lfun{substitutions}\ops{=}{[}{]}{)}
\end{Verbatim}



\end{document}